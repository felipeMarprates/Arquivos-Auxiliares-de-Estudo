
\documentclass[11pt]{article}
\usepackage[T1]{fontenc}
\usepackage[utf8]{inputenc}
\usepackage[portuguese]{babel}
\usepackage{amsmath}
\usepackage{amssymb}
\usepackage{amsthm}
\usepackage{geometry}
\geometry{a4paper, margin=0.5in, landscape}
\usepackage{multicol}
\usepackage{enumitem}
\usepackage{xcolor} % <— essencial para \colorbox
\usepackage{comment} % Include the comment package

\setlength{\columnsep}{12pt}
\raggedcolumns
\pagenumbering{gobble} % Remove os números de página
\newtheoremstyle{yellowhead}% name
{6pt}{3pt}% Space above/below
{}% body font
{}% indent
{\bfseries}% head font
{}% punctuation after head
{0.2em}% space after head
{\colorbox{yellow!30}{\strut #1\ifx#2\empty\else~#2\fi\ifx#3\empty\else~\ (#3)\fi}}% head spec

% Ativa o estilo e (re)declara seus ambientes
\theoremstyle{yellowhead}
\newtheorem*{theorem}{Teorema}
\newtheorem*{lemma}{Lema}
\newtheorem*{proposition}{Prop.}
\newtheorem*{corollary}{Cor.}

% Definições/observações normalmente usam corpo em romano:
\newtheoremstyle{yellowdef}
{6pt}{6pt}{\normalfont}{}{\bfseries}{.}{0.6em}
{\colorbox{yellow!30}{\strut #1\ifx#2\empty\else~#2\fi\ifx#3\empty\else~\ (#3)\fi}}
\theoremstyle{yellowdef}
\newtheorem*{definition}{Def.}
\newtheorem{exemplo}{Exemplo}
\newtheorem{observacao}{Observação}

\begin{document}

\title{Sumário de Teoremas, Proposições e Lemas Fundamentais em Análise Funcional}
\author{}
\date{}

\section{Desigualdades Fundamentais para Espaços Normados}

\begin{multicols}{3}

\begin{definition}[Espaço Vetorial]
Um **espaço vetorial** é um conjunto não vazio $V$ sobre um corpo $K$ ($K = \mathbb{R}$ ou $\mathbb{C}$) munido de duas operações, adição ($+: V \times V \rightarrow V$) e multiplicação por escalar ($\cdot: \mathbb{R} \times V \rightarrow V$), que satisfaz as seguintes propriedades: (a) $u + (v + w) = (u + v) + w$ (associativa), (b) $u + v = v + u$ (comutativa), (c) Existe $0 \in V$ tal que $0 + u = u$ para todo $u \in V$, (d) Para cada $v \in V$ existe $-v \in V$ tal que $v + (-v) = 0$, (e) $\alpha \cdot (\beta \cdot u) = (\alpha\beta) \cdot v$, para todo $\alpha, \beta \in K$ e $u \in V$, (f) $1 \cdot u = u$, para todo $y \in V$, (g) $\alpha \cdot (u + v) = \alpha \cdot u + \alpha \cdot v$, para todo $\alpha, \beta \in K$ e $v \in V$, (h) $(\alpha + \beta)u = \alpha \cdot u + \beta \cdot v$, para todo $\alpha, \beta \in K$ e $v \in V$.
\end{definition}

\begin{definition}[Norma]
Seja $V$ um espaço vetorial sobre o corpo $K$ ($K = \mathbb{R}$ ou $\mathbb{C}$). Uma **norma** em $V$ é uma função $\Vert \cdot \Vert_V : V \rightarrow \mathbb{R}^+$ que satisfaz as seguintes propriedades: $\Vert v \Vert_V = 0 \Leftrightarrow v = 0$, $\Vert \lambda v \Vert_V = |\lambda|\Vert v \Vert_V$ e $\Vert v + w \Vert_V \le \Vert v \Vert_V + \Vert w \Vert_V$, para todo $v, w \in V$.
\end{definition}

\begin{definition}[Espaço Vetorial Normado]
Um espaço vetorial $V$ munido de uma norma é chamado **espaço vetorial normado**, isto é, um espaço vetorial normado é um par $(V, \Vert \cdot \Vert_V)$ onde $\Vert \cdot \Vert_V : V \rightarrow \mathbb{R}^+$ é uma norma.
\end{definition}

\begin{definition}[Métrica de Espaço Métrico]
Seja $X$ um conjunto não vazio. Uma **métrica** ou **distância** em $X$ é uma função $\rho : X \times X \rightarrow [0, \infty)$ satisfazendo: $\rho(x, y) = 0 \Leftrightarrow x = y$, $\rho(x, y) = \rho(y, x)$, para todo $x, y \in X$, $\rho(x, z) \le \rho(x, y) + \rho(y, z)$, para todo $x, y, z \in X$. O conjunto $X$ munido da métrica $\rho$ é chamado **espaço métrico** e é denotado por $(X, \rho)$.
\end{definition}

Essas desigualdades são ferramentas essenciais para provar a desigualdade triangular para as p-normas, garantindo a validade dessas normas.

\begin{lemma}[Desigualdade de Young]
Se $p \in (1, \infty)$, $q \in (1, \infty)$ é tal que $\frac{1}{p} + \frac{1}{q} = 1$ e $a, b \in [0, \infty)$, então $a^{1/p} b^{1/q} \le \frac{a}{p} + \frac{b}{q}$.
\end{lemma}

\begin{lemma}[Desigualdade de Hölder]
Se $p \in (1, \infty)$ e $q \in (1, \infty)$ é tal que $\frac{1}{p} + \frac{1}{q} = 1$, então
\[
\sum_{i=1}^{N} |x_i y_i| \le \left[ \sum_{i=1}^{N} |x_i|^p \right]^{1/p} \left[ \sum_{i=1}^{N} |y_i|^q \right]^{1/q}
\]
para todo $x = (x_1, \dots, x_N), y = (y_1, \dots, y_N) \in \mathbb{R}^N$.
\end{lemma}

\begin{lemma}[Desigualdade de Minkowski]
Se $p \in [1, \infty]$, então
\[
\left[ \sum_{i=1}^{N} |x_i + y_i|^p \right]^{1/p} \le \left[ \sum_{i=1}^{N} |x_i|^p \right]^{1/p} + \left[ \sum_{i=1}^{N} |y_i|^p \right]^{1/p}
\]
ou, em outras palavras, $\Vert x + y \Vert_p \le \Vert x \Vert_p + \Vert y \Vert_p$ para todo $x = (x_1, \dots, x_N), y = (y_1, \dots, y_N) \in \mathbb{R}^N$.
\end{lemma}

Essas desigualdades ajudam a estabelecer as propriedades fundamentais dos espaços normados, o que leva naturalmente ao estudo de sua completude e do comportamento das transformações entre eles.
\end{multicols}

\section{Completude e Transformações Lineares}
\begin{multicols}{3}
\begin{definition}[Espaço de Banach]
Um espaço vetorial normado que é **completo** com a métrica induzida pela norma é dito um **espaço de Banach**.
\end{definition}

\begin{definition}[Normas Equivalentes]
Duas normas em $X$, $\Vert \cdot \Vert_1$ e $\Vert \cdot \Vert_2$, são **equivalentes** se existem constantes positivas $c_1$ e $c_2$ tal que $c_1\Vert x \Vert_1 \le \Vert x \Vert_2 \le c_2\Vert x \Vert_1 \forall x \in X$.
\end{definition}

\begin{definition}[Série Convergente e Abs. Convergente]
Uma série $\sum_{j=1}^{\infty} x_j$ é dita **convergente** em $X$ se a sequência de somas parciais $\sum_{j=1}^{n} x_j$ converge para $x$ quando $n \rightarrow \infty$. Uma série é dita **absolutamente convergente** se a série das normas $\sum_{j=1}^{\infty} \Vert x_j \Vert$ é convergente.
\end{definition}

\begin{definition}[Transformação Linear Limitada]
Uma transformação linear $T : X \rightarrow Y$ entre dois espaços vetoriais normados é **limitada** se existe uma constante $c \ge 0$ tal que $\Vert Tx \Vert_Y \le c\Vert x \Vert_X$, para todo $x \in X$.
\end{definition}

\begin{definition}[Transformação Linear Inversível]
$T \in L(X ,Y )$ é **inversível** ou um **isomorfismo** se $T$ é bijetora e $T^{-1} \in L(Y ,X )$, o que é equivalente a existir $c > 0$ tal que $\Vert Tx \Vert_Y \ge c\Vert x \Vert_X$.
\end{definition}

\begin{definition}[Isometria (Transformação Linear)]
$T$ é uma **isometria** se $\Vert Tx \Vert_Y = \Vert x \Vert_X$, para todo $x \in X$.
\end{definition}

\begin{theorem}[1 Condição para a Completude]
Um espaço vetorial normado é completo se, e somente se, toda série absolutamente convergente é convergente.
\end{theorem}

\begin{proposition}[Equiv. p. Transf. Lineares Contínuas]
Se $X, Y$ são espaços vetoriais normados e $T: X \to Y$ é linear, as seguintes afirmações são equivalentes:
\begin{enumerate}[label=(\alph*)]
    \item $T$ é contínua;
    \item $T$ é contínua em $0$;
    \item $T$ é limitada.
\end{enumerate}
\end{proposition}

\begin{proposition}[Completude de $L(X, Y)$]
Se $Y$ é completo, então $L(X, Y)$ é completo.
\end{proposition}
O conceito de transformações lineares limitadas entre espaços completos (espaços de Banach) leva diretamente ao estudo mais detalhado de funcionais lineares, que são centrais para os teoremas de Hahn-Banach.
\end{multicols}

\section{Os Teoremas de Hahn-Banach e Suas Consequências}
\begin{multicols}{3}
\begin{definition}[Funcional Linear]
Seja $X$ um espaço vetorial sobre $K$. Uma função linear $f : X \rightarrow K$ é chamada um **funcional linear**.
\end{definition}

\begin{definition}[Espaço Dual ($X^*$)]
Se $X$ é um espaço vetorial normado, $L(X ,K)$ é um espaço de Banach que é chamado **espaço dual de $X$** e denotado por $X^*$.
\end{definition}

\begin{definition}[Funcional Sublinear]
Se $X$ é normado, um **funcional sublinear** é uma função $p : X \rightarrow \mathbb{R}$ tal que $p(x+y) \le p(x) + p(y)$ e $p(\lambda x) = \lambda p(x)$, para todo $x, y \in X$ e $\lambda > 0$.
\end{definition}

\begin{definition}[Espaço Reflexivo]
Para espaços de dimensão infinita, quando a imagem isométrica $\hat{X}$ é igual ao bidual $X^{**}$, $X$ é dito **reflexivo**. A reflexividade passa a ser entendida como $X = X^{**}$ quando $X$ é identificado com seu subespaço $\hat{X}$ no bidual $X^{**}$.
\end{definition}

\begin{definition}[Hiperplano Afim]
Um **hiperplano (afim)** é um conjunto da forma $H = \{x \in X : f(x) = \alpha\}$ onde $f : X \rightarrow \mathbb{R}$ é um funcional linear não identicamente nulo e $\alpha \in \mathbb{R}$. Diremos que $H$ é o hiperplano de equação $[f = \alpha]$.
\end{definition}

\begin{definition}[Conjunto Convexo]
Seja $X$ um espaço vetorial sobre $K$. Diremos que $C \subset X$ é **convexo** se $tx + (1-t)y \in C$ sempre que $t \in$ e $x, y \in C$.
\end{definition}

\begin{definition}[Separação Fraca]
Se $A, B \subset X$ dizemos que o hiperplano de equação $[f = \alpha]$ separa $A$ e $B$ no **sentido fraco** se $f(x) \le \alpha$ para todo $x \in A$ e $f(x) \ge \alpha$ para todo $x \in B$.
\end{definition}

\begin{definition}[Separação Forte]
Diremos que o hiperplano de equação $[f = \alpha]$ separa $A$ e $B$ no **sentido forte** se existe $\varepsilon > 0$ tal que $f(x) \le \alpha - \varepsilon$ para todo $x \in A$ e $f(x) \ge \alpha + \varepsilon$ para todo $x \in B$.
\end{definition}

\begin{definition}[Func. de Minkowski de um Convexo]
Seja $X$ um espaço vetorial normado sobre $\mathbb{R}$ e $C \subset X$ um aberto convexo com $0 \in C$. Para todo $x \in X$, o **funcional de Minkowski de $C$**, $p(x)$, é definido como $p(x) = \inf\{\alpha > 0 ; \alpha^{-1}x \in C\}$.
\end{definition}

\begin{definition}[Complexificação de X ($X^{\#}$)]
Seja $X$ um espaço vetorial sobre $\mathbb{R}$. A **complexificação de $X$** estende $X$ a um espaço vetorial $X^{\#}$ sobre $\mathbb{C}$ e é feita da seguinte forma: $X^{\#} = \{(x, y) : x, y \in X\}$ com a operação de adição coordenada a coordenada e com a operação de multiplicação por escalar dada por $(\alpha + i\beta)(x, y) = (\alpha x - \beta y, \beta x + \alpha y)$.
\end{definition}
\begin{proposition}[Relação entre Func. Lineares $\mathbb{R}$ e $\mathbb{C}$]
Seja $X$ um espaço vetorial sobre $\mathbb{C}$. Se $f: X \to \mathbb{C}$ é um funcional linear e $u = \text{Re}(f)$, então $u$ é um funcional linear real e $f(x) = u(x) - iu(ix)$ para todo $x \in X$.
Reciprocamente, se $u: X \to \mathbb{R}$ é um funcional linear real e $f: X \to \mathbb{C}$ é definido por $f(x) = u(x) - iu(ix)$, então $f$ é um funcional linear complexo. Se $X$ é normado, $f$ é limitado se e somente se $u$ é limitado, e neste caso $\Vert f \Vert = \Vert u \Vert$.
\end{proposition}

\begin{theorem}[Hahn-Banach (E.V.$\mathbb{R}$)]
Sejam $X$ um espaço vetorial real, $p$ um funcional sublinear em $X$, $M$ um subespaço vetorial de $X$ e $f$ um funcional linear em $M$ tal que $f(x) \le p(x)$ para todo $x \in M$. Então existe um funcional linear $F$ em $X$ tal que $F(x) \le p(x)$ para todo $x \in X$ e $F|_M = f$.
\end{theorem}

\begin{theorem}[Hahn-Banach (E.V.$\mathbb{C}$)]
Sejam $X$ um espaço vetorial complexo, $p$ uma seminorma em $X$, $M$ um subespaço vetorial de $X$ e $f: M \to \mathbb{C}$ um funcional linear com $|f(x)| \le p(x)$ para $x \in M$.
Então existe $F: X \to \mathbb{C}$, um funcional linear, tal que $|F(x)| \le p(x)$ para todo $x \in X$ e $F|_M = f$.
\end{theorem}

\begin{corollary}[de Hahn-Banach]
Seja $X$ um espaço vetorial normado.
\begin{enumerate}
    \item Se $M$ é um subespaço vetorial fechado de $X$ e $x \in X \setminus M$, existe $f \in X^*$ tal que $f(x) \neq 0$ e $f|_M = 0$. Na verdade, se $\delta = \inf_{y \in M} \Vert x - y \Vert$, $f$ pode ser escolhido tal que $\Vert f \Vert = 1$ e $f(x) = \delta$.
    \item Se $x \neq 0$, existe $f \in X^*$ tal que $\Vert f \Vert = 1$ e $f(x) = \Vert x \Vert$.
    \item Os funcionais lineares limitados em $X$ separam pontos.
    \item Se $x \in X$, defina $\hat{x}: X^* \to \mathbb{C}$ por \mbox{$\hat{x}(f) = f(x)$}, $\forall f \in X^*$. Então a transformação $x \to \hat{x}$ é uma isometria linear de $X$ em $X^{**}$.
\end{enumerate}
\end{corollary}

\begin{proposition}[Hiperplanos Fechados]
O hiperplano com equação $[f = \alpha]$ é fechado se e somente se $f$ é contínuo.
\end{proposition}

\begin{lemma}[O Funcional de Minkowski]
Seja $X$ um espaço vetorial normado sobre $\mathbb{R}$ e $C \subset X$ um conjunto convexo aberto com $0 \in C$. Para cada $x \in X$ defina $p(x) = \inf\{\alpha > 0 : \alpha^{-1}x \in C\}$ ($p$ é o funcional de Minkowski de $C$). Então, $p$ é um funcional sublinear e existe $M$ tal que $0 \le p(x) \le M\Vert x \Vert$, $\forall x \in X$, e $C = \{x \in X: p(x) < 1\}$.
\end{lemma}

\begin{lemma}[Separação Ponto de um Convexo]
Sejam $C \subset X$ um conjunto convexo aberto e não vazio e $x_0 \in X \setminus C$. Então existe $f \in X^*$ tal que $f(x) < f(x_0)$ para todo $x \in C$. Em particular, o hiperplano fechado com equação $[f = f(x_0)]$ separa fracamente $C$ de $x_0$.
\end{lemma}

\begin{theorem}[1 Forma Geométrica Hahn-Banach]
Sejam $X$ um espaço vetorial normado real e $A, B \subset X$ dois conjuntos convexos, não vazios e disjuntos. Se $A$ é aberto, existe um hiperplano fechado que separa fracamente $A$ e $B$.
\end{theorem}

\begin{theorem}[2 Forma Geométrica Hahn-Banach]
Sejam $X$ um espaço vetorial normado real, e $A$ e $B$ conjuntos convexos, não vazios e disjuntos em $X$. Suponha que $A$ é fechado e $B$ é compacto. Então existe um hiperplano fechado que separa fortemente $A$ e $B$.
\end{theorem}

\begin{corollary}[Anuladores para Subespaços Próprios]
Sejam $X$ um espaço vetorial normado sobre $\mathbb{K}$ e $F \subset X$ um subespaço vetorial próprio de $X$ ($\bar{F} \neq X$). Então, existe $f \in X^*$, $f \neq 0$ tal que $f(x) = 0$, $\forall x \in F$.
\end{corollary}
\end{multicols}

\section{Consequências do Teorema da Categoria de Baire}
\begin{multicols}{3}
    \begin{definition}[Auto-valor e Auto-vetor]
Considere um espaço vetorial $X$ sobre o corpo $K$ e uma transformação linear $A : X \rightarrow X$. Se a transformação linear $\lambda I - A$ (para cada escalar $\lambda \in K$) não é injetiva, existe $0 \ne x \in X$ tal que $(\lambda I - A)x = 0$. Neste caso, diremos que $\lambda$ é um **auto-valor de $A$** e que $x$ é um **auto-vetor de $A$ associado ao auto-valor $\lambda$**.
\end{definition}

\begin{definition}[Produto Escalar (Produto Interno)]
Seja $H$ um espaço vetorial sobre $K$. Um **produto escalar** em $H$ é uma função $\langle \cdot, \cdot \rangle : H \times H \rightarrow K$ tal que: (a) $\langle u, v \rangle = \overline{\langle v, u \rangle}$ para todo $u, v \in H$, (b) $\langle \alpha u + \beta v, w \rangle = \alpha \langle u, w \rangle + \beta \langle v, w \rangle$, para todo $u, v, w \in H, \alpha, \beta \in K$, (c) $\langle u, u \rangle \ge 0$ e $\langle u, u \rangle = 0$ se, e somente se, $u = 0$.
\end{definition}

\begin{definition}[Vetores Ortogonais]
Dois vetores $u, v$ em um espaço com produto interno $H$ são ditos **ortogonais** (escrevemos $u \perp v$) se $\langle u, v \rangle = 0$.
\end{definition}

\begin{definition}[Projeção sobre o Convexo ($P_K$)]
Se $K$ é um subconjunto fechado e convexo de um espaço de Hilbert $H$ e $u_0 \in H$, existe um único $v_0 \in K$ tal que $\Vert u_0 - v_0 \Vert = \inf_{v \in K} \Vert u_0 - v \Vert$. Escrevemos $v_0 = P_K u_0$ e dizemos que $P_K$ é a **projeção sobre o convexo $K$**.
\end{definition}

\begin{definition}[Projeção (Transformação Linear)]
Uma transformação linear $P : H \rightarrow M$ é dita uma **projeção** se $P^2 = P$.
\end{definition}

\begin{definition}[Projeção Ortogonal]
Se $P \in L(H)$ é uma projeção, $M = Im(P)$ e $M^{\perp} = N(P)$ dizemos que $P$ é uma **projeção ortogonal sobre $M$**.
\end{definition}

\begin{definition}[Conjunto Ortonormal]
Um subconjunto $\{u_{\alpha}\}_{\alpha \in A}$ de $H$ é chamado um **conjunto ortonormal** se $\Vert u_{\alpha} \Vert = 1$ para todo $\alpha \in A$ e $u_{\alpha} \perp u_{\beta}$ para $\alpha \ne \beta$.
\end{definition}

\begin{definition}[Base Ortonormal]
Um conjunto ortonormal tendo as propriedades (a-c) do Teorema 6 (ou Teorema 2), que são Completamento, Identidade de Parseval e Convergência de Série, é chamado uma **base ortonormal de $H$**.
\end{definition}

\begin{definition}[Transformação Unitária]
Se $H_1$ e $H_2$ são espaços de Hilbert com produtos escalares $\langle \cdot, \cdot \rangle_1, \langle \cdot, \cdot \rangle_2$, uma **transformação unitária** de $H_1$ sobre $H_2$ é uma transformação linear sobrejetora $U : H_1 \rightarrow H_2$ que preserva produto escalar; isto é, $\langle Ux, Uy \rangle_2 = \langle x, y \rangle_1$.
\end{definition}

\begin{definition}[T.L. Densamente Definida]
Se $D(A)$ (o domínio de $A$) é denso em $X$, dizemos que $A$ é **densamente definida**.
\end{definition}

\begin{definition}[(Domínio, Gráfico, Imagem, Núcleo)]
Seja $A : D(A) \subset X \rightarrow Y$ uma transformação linear. Então, $D(A)$ é o **domínio de $A$**; $G(A) = \{(x ,Ax) \in X \times Y : u \in D(A)\} \subset X \times Y$ é o **Gráfico de $A$**; $Im(A) = \{Ax \in Y : x \in D(A)\} \subset Y$ é a **Imagem de $A$**; e $N(A) = \{x \in D(A) : Ax = 0\}$ é o **Núcleo de $A$**.
\end{definition}

\begin{definition}[Transformação Linear Fechada]
Diremos que uma transformação linear $T$ é **fechada** se o seu gráfico $G(T) = \{(x, Tx) : x \in X\}$ for fechado em $X \times Y$. É equivalente a dizer que, para toda seqüência $\{(u_n, Au_n)\}$ em $D(A) \times Y$ que é convergente em $X \times Y$ para $(u, v) \in X \times Y$, temos que $u \in D(A)$ e $Au = v$.
\end{definition}

\begin{definition}[Transformação Linear Fechável]
Diremos que uma transformação linear $A$ é **fechável** se $G(\overline{A})$ é gráfico de uma transformação linear. É equivalente a dizer que, sempre que uma seqüência $\{(u_n, Au_n)\}$ em $D(A) \times Y$ converge, em $X \times Y$, para $(0, v) \in X \times Y$, temos que $v = 0$.
\end{definition}

\begin{definition}[Conjunto Nunca Denso (ou Raro)]
Se $(X , \rho)$ for um espaço métrico, um conjunto $A \subset X$ será **nunca denso** ou **raro** se o seu fecho tiver interior vazio.
\end{definition}

\begin{definition}[Conjunto de Primeira Categoria]
Um conjunto $A \subset X$ será de **Primeira Categoria** em $X$ se for união enumerável de conjuntos nunca densos.
\end{definition}

\begin{definition}[Conjunto de Segunda Categoria]
Um conjunto será de **Segunda Categoria** em $X$ se não for de Primeira Categoria.
\end{definition}

\begin{definition}[Aplicação Aberta]
Sejam $X ,Y$ espaços vetoriais normados e $T : X \rightarrow Y$ uma transformação linear. Diremos que $T$ será **aberta** se $T(U)$ for aberto em $Y$, sempre que $U$ for aberto em $X$.
\end{definition}
\begin{proposition}[Propriedade de Esp. de 2 Categoria]
Um espaço $(X, \rho)$ será de segunda categoria em si mesmo se, e somente se, em qualquer representação de $X$ como uma união contável de conjuntos fechados, pelo menos um deles contém uma bola aberta.
\end{proposition}

\begin{theorem}[Categoria de Baire]
Todo espaço métrico completo é de segunda categoria em si mesmo.
\end{theorem}

\begin{corollary}[Esp. de Banach são de 2 Categoria]
Todo espaço de Banach é de segunda categoria em si mesmo.
\end{corollary}

\begin{theorem}[Mapeamento Aberto]
Seja $X$ um espaço de Banach e $Y$ um espaço vetorial normado. Se $T \in L(X, Y)$ e $T(X)$ é de segunda categoria em $Y$, então:
\begin{enumerate}[label=(\alph*)]
    \item $T$ será sobrejetor;
    \item $T$ será um mapeamento aberto; e
    \item $Y$ será de segunda categoria.
\end{enumerate}
\end{theorem}

\begin{lemma}[Equivalência de Mapeamento Aberto]
Sejam $X, Y$ espaços vetoriais normados e $T: X \to Y$ uma transformação linear. As seguintes afirmações são equivalentes:
\begin{enumerate}[label=(\alph*)]
    \item $T$ é um mapeamento aberto;
    \item Existe $r > 0$ tal que $T(B_1^X(0)) \supset B_r^Y(0)$.
\end{enumerate}
\end{lemma}

\begin{lemma}[Condição p. Inclusão da Imagem]
Se $X$ é um espaço de Banach, $Y$ é um espaço vetorial normado e $T \in L(X, Y)$ é tal que, para algum $r > 0$, $B_r^Y(0) \subset [T(B_1^X(0))]^-$, então $B_{r/2}^Y(0) \subset T(B_1^X(0))$.
\end{lemma}

\begin{corollary}[do Teorema do Mapeamento Aberto]
Sejam $X$ e $Y$ espaços de Banach.
\begin{enumerate}[label=(\alph*)]
    \item Se $T \in L(X, Y)$ é sobrejetor, então $T$ é aberto.
    \item Se $T \in L(X, Y)$ é bijetor, então $T$ é um isomorfismo.
\end{enumerate}
\end{corollary}

\begin{theorem}[Princípio da Limitação Uniforme]
Sejam $X$ e $Y$ espaços vetoriais normados e $A \subset L(X, Y)$.
\begin{enumerate}[label=(\alph*)]
    \item Se $\{x \in X: \sup\{\Vert Tx \Vert : T \in A\} < \infty\}$ é de segunda categoria, então $\sup\{\Vert T \Vert : T \in A\} < \infty$.
    \item Se $X$ é um espaço de Banach e $\{x \in X: \sup\{\Vert Tx \Vert : T \in A\} < \infty\} = X$, então $\sup\{\Vert T \Vert : T \in A\} < \infty$.
    \item Se $X$ é um espaço de Banach, $\{T_n : n \in \mathbb{N}\} \subset L(X, Y)$, $\{T_nx\}$ é convergente para cada $x \in X$, e $T: X \to Y$ é definido por $Tx = \lim_{n \to \infty} T_nx$, então $T \in L(X, Y)$ e $\Vert T \Vert \le \liminf \Vert T_n \Vert$.
\end{enumerate}
\end{theorem}

\begin{corollary}[Limitação de Subconjuntos em $X$]
Se $X$ é um espaço de Banach, $B \subset X$, e $f(B) = \{f(b): b \in B\}$ é limitado para todo $f \in X^*$, então $B$ é limitado.
\end{corollary}

\begin{corollary}[Limitação de Subconjuntos em $X^*$]
Seja $X$ um espaço de Banach e $B^* \subset X^*$. Suponha que para todo $x \in X$ o conjunto $B^*(x) = \{b^*(x): b^* \in B^*\}$ é limitado. Então $B^*$ é limitado.
\end{corollary}

\begin{theorem}[Gráfico Fechado]
Se $X$ e $Y$ são espaços de Banach e $T: X \to Y$ é fechada, então $T$ é limitada.
\end{theorem}
\end{multicols}
\newpage
\section{Espaços de Hilbert: Projeções, Representação e Bases}
\begin{multicols}{3}

\begin{definition}[Espaço com Produto Interno]
Um espaço vetorial $H$ juntamente com um produto interno é dito um **espaço com produto interno**.
\end{definition}

\begin{definition}[Espaço de Hilbert]
Se um espaço com produto interno $H$ é completo dizemos que $H$ é um **espaço de Hilbert**.
\end{definition}

\begin{lemma}[Projeção em um Convexo Fechado]
Se $K$ é um subconjunto fechado e convexo de um espaço de Hilbert $H$ e $u_0 \in H$, existe um único $v_0 \in K$ tal que $\Vert u_0 - v_0 \Vert = \inf_{v \in K} \Vert u_0 - v \Vert$.
\end{lemma}

\begin{proposition}[Caract. do Operacão de Projeção]
Seja $H$ um espaço de Hilbert, $K \subset H$ fechado e convexo, e $u_0 \in H$. Então $\text{Re}\langle u_0 - P_K u_0, w - P_K u_0 \rangle \le 0$, para todo $w \in K$.
\end{proposition}

\begin{proposition}[Caract. Conversa da Projeção]
Seja $H$ um espaço de Hilbert e $K \subset H$ um conjunto convexo fechado e não vazio. Se, dado $u_0 \in H$, existe $v_0 \in K$ tal que $\text{Re}\langle u_0 - v_0, w - v_0 \rangle \le 0$ para todo $w \in K$, então $v_0 = P_K u_0$.
\end{proposition}

\begin{corollary}[Caract. e Linearidade da Projeção ]
Se $H$ é um espaço de Hilbert e $M$ é um subespaço vetorial fechado de $H$, então $P_M: H \to H$ é caracterizado por $v = P_M u$ se, e somente se, $\langle u - v, w \rangle = 0$, para todo $w \in M$. Disso se segue que $P_M$ é linear e $P_M^2 = P_M$.
\end{corollary}

\begin{theorem}[Propriedades do Op. de Projeção]
Se $H$ é um espaço de Hilbert e $K \subset H$ é um conjunto convexo fechado, então $\Vert P_K u_1 - P_K u_2 \Vert \le \Vert u_1 - u_2 \Vert$, para todos $u_1, u_2 \in H$.
\end{theorem}

\begin{theorem}[Decomposição Ortogonal]
Seja $H$ um espaço de Hilbert e $M$ um subespaço vetorial fechado de $H$. Então $M \oplus M^\perp = H$; ou seja, cada $u \in H$ pode ser unicamente expresso como $u = w + v$, onde $w \in M$ e $v \in M^\perp$.
\end{theorem}

\begin{theorem}[Representação de Riesz]
Se $f \in H^*$, existe um único $y \in H$ tal que $f(x) = \langle x, y \rangle$ para todo $x \in H$.
\end{theorem}

\begin{theorem}[A Desigualdade de Bessel]
Se $\{u_\alpha\}_{\alpha \in A}$ é um conjunto ortonormal em $H$, então para $u \in H$,
\[
\sum_{\alpha \in A} |\langle u, u_\alpha \rangle|^2 \le \Vert u \Vert^2.
\]
Em particular, $\{\alpha \in A : \langle u, u_\alpha \rangle \neq 0\}$ é enumerável.
\end{theorem}

\begin{theorem}[Equiv. p. Base Ortonormal]
Se $\{u_\alpha\}_{\alpha \in A}$ é um conjunto ortonormal em $H$, as seguintes afirmações são equivalentes:
\begin{enumerate}[label=(\alph*)]
    \item (Completude) Se $\langle u, u_\alpha \rangle = 0$ para todo $\alpha \in A$, então $u = 0$.
    \item (Identidade de Parseval) Para todo $u \in H$,
    \[
    \Vert u \Vert^2 = \sum_{\alpha \in A} |\langle u, u_\alpha \rangle|^2.
    \]
    \item Para cada $u \in H$,
    \[
    u = \sum_{\alpha \in A} \langle u, u_\alpha \rangle u_\alpha,
    \]
    onde a soma tem apenas um número contável de termos não nulos e converge independentemente da ordem dos termos.
\end{enumerate}
\end{theorem}

\begin{proposition}[Existência de uma Base Ortonormal]
Todo espaço de Hilbert tem uma base ortonormal.
\end{proposition}

\begin{theorem}[Separabilidade e Bases]
Um espaço de Hilbert $H$ é separável se e somente se ele tem uma base ortonormal enumerável, e neste caso, toda base ortonormal de $H$ é enumerável.
\end{theorem}

\begin{proposition}[Transformação Unitária para $l^2(A)$]
Seja $\{u_\alpha\}_{\alpha \in A}$ uma base ortonormal de $H$. Então a correspondência $x \to \hat{x}$ definida por $\hat{x}(\alpha) = \langle x, u_\alpha \rangle$ é uma transformação unitária de $H$ para $l^2(A)$.
\end{proposition}
\end{multicols}

\begin{comment}

\section{Operadores Duais e Anuladores}
\begin{multicols}{3}
\begin{definition}[Operador Dual ($A^*$)]
Se $A : D(A) \subset X \rightarrow Y$ é uma transformação linear densamente definida, o **dual** $A^* : D(A^*) \subset Y^* \rightarrow X^*$ de $A$ é o operador linear definido por: $D(A^*) = \{y^* \in Y^* : y^* \circ A : D(A) \subset X \rightarrow K \text{ é limitado }\}$, ou equivalentemente, $D(A^*) = \{y^* \in Y^*:\exists z^* \in X^* \text{ tal que } \langle Ax , y^*\rangle=\langle x , z^*\rangle, \forall x \in D(A)\}$. Se $y^* \in D(A^*)$, define-se $A^*y^* := z^*$.
\end{definition}

\begin{definition}[Anulador]
Seja $X$ um espaço vetorial normado sobre $K$. Para um subespaço vetorial $M \subset X$, o **anulador de $M$** é $M^{\perp} = \{f \in X^* : f(x) = 0, \forall x \in M\}$. Para um subespaço vetorial $N^* \subset X^*$, o **anulador de $N^*$** é $(N^*)^{\perp} = \{x \in X : f(x) = 0, \forall f \in N\}$.
\end{definition}

\begin{definition}[Conjunto Total]
Um subconjunto $M^* \subset X^*$ é dito **total** se $(M^*)^{\perp} = \{0\}$.
\end{definition}
\begin{proposition}[O Operador Dual é Fechado]
Se $X$ e $Y$ são espaços vetoriais normados e $A: D(A) \subset X \to Y$ é linear e densamente definido, então $A^*: D(A^*) \subset Y^* \to X^*$ é um operador linear fechado.
\end{proposition}

\begin{lemma}[A Norma do Operador Dual]
Sejam $X$ e $Y$ espaços de Banach sobre $\mathbb{K}$ e $A \in L(X, Y)$; então, $A^* \in L(Y^*, X^*)$ e $\Vert A \Vert_{L(X, Y)} = \Vert A^* \Vert_{L(Y^*, X^*)}$.
\end{lemma}

\begin{lemma}[Densidade do Domínio do Op. Dual]
Seja $Y$ um espaço de Banach reflexivo sobre $\mathbb{K}$. Se $A: D(A) \subset X \to Y$ é fechado e densamente definido, então $D(A^*)$ é denso em $Y^*$.
\end{lemma}

\begin{theorem}[O Inverso de um Operador Dual]
Se $S: D(S) \subset X \to X$ é um operador linear injetivo, densamente definido com imagem densa, então $S^*: D(S^*) \subset X^* \to X^*$ é bem-definido, injetivo, e $(S^*)^{-1} = (S^{-1})^*$.
\end{theorem}

\begin{proposition}[Propriedades de Anuladores]
Seja $X$ um espaço vetorial normado sobre $\mathbb{K}$.
\begin{enumerate}
    \item Se $M$ é um subespaço vetorial de $X$, então $(M^\perp)^\perp = \bar{M}$.
    \item Se $N^*$ é um subespaço vetorial de $X^*$, então $((N^*)^\perp)^\perp \supset \bar{N^*}$ (com a igualdade valendo se $X$ é reflexivo).
\end{enumerate}
\end{proposition}

\begin{lemma}[Anuladores e Inclusão de Sub]
Seja $X$ um espaço vetorial normado sobre $\mathbb{K}$, $M_1, M_2$ subespaços vetoriais de $X$ com $M_1 \subset M_2$, e $N_1^*, N_2^*$ subespaços vetoriais de $X^*$ com $N_1^* \subset N_2^*$. Então $M_2^\perp \subset M_1^\perp$ e $(N_2^*)^\perp \subset (N_1^*)^\perp$.
\end{lemma}

\begin{proposition}[Gráfico do Op. Dual como Anulador]
Seja $A: D(A) \subset X \to X$ um operador linear densamente definido. O gráfico de $A^*$, $G(A^*) = \{(x^*, A^*x^*) : x^* \in D(A^*)\}$, é o anulador em $X^* \times X^*$ de $\{(-Ax, x) : x \in D(A)\}$.
\end{proposition}

\begin{proposition}[Domínio do Operador Dual é Total]
Se $A: D(A) \subset X \to X$ é fechado e densamente definido, então $D(A^*)$ é total.
\end{proposition}
\end{multicols}



\section{Operadores Compactos}
\begin{multicols}{3}
\begin{definition}[Operador Compacto]
Sejam $X ,Y$ espaços de Banach sobre $K$. Diremos que um operador linear $K : X \rightarrow Y$ é **compacto** se $K (B_X^1(0))$ é um subconjunto relativamente compacto de $Y$.
\end{definition}

\begin{definition}[Operador Adjunto ($A^{\bullet}$)]
Se $H$ é um espaço de Hilbert e $A : D(A) \subset H \rightarrow H$ é um operador densamente definido, o **adjunto** $A^{\bullet} : D(A^{\bullet}) \subset H \rightarrow H$ de $A$ é definido por $D(A^{\bullet}) = \{u \in H : v \mapsto \langle Av , u\rangle_H : D(A)\rightarrow K \text{ é limitado}\}$. Se $u \in D(A^{\bullet})$, $A^{\bullet}u$ é o único elemento de $H$ tal que $\langle v ,A^{\bullet}u\rangle_H = \langle Av , u\rangle_H$, para todo $v \in D(A)$.
\end{definition}

\begin{definition}[Operador Simétrico (ou Hermitiano)]
Seja $H$ um espaço de Hilbert sobre $K$ com produto interno $\langle \cdot, \cdot \rangle_H$. Diremos que um operador $A : D(A) \subset H \rightarrow H$ é **simétrico** (também chamado **Hermitiano** quando $K = \mathbb{C}$) se $D(A) = H$ e $A \subset A^*$; isto é, $\langle Ax, y \rangle_H = \langle x, Ay \rangle_H$ para todo $x, y \in D(A)$.
\end{definition}

\begin{definition}[Operador Auto-adjunto]
Diremos que um operador $A$ é **auto-adjunto** se $A = A^*$.
\end{definition}
\begin{theorem}[Esp. Op. Compactos é Fechado]
Sejam $X, Y$ espaços de Banach sobre $\mathbb{K}$. Então $K(X, Y)$ é um subespaço fechado de $L(X, Y)$.
\end{theorem}

\begin{theorem}[Propriedades de Op. Compactos]
Sejam $X, Y, Z$ espaços de Banach sobre um corpo $\mathbb{K}$, $A \in L(X, Y)$ e $B \in L(Y, Z)$.
\begin{enumerate}[label=(\alph*)]
    \item Se $A \in K(X, Y)$ ou $B \in K(Y, Z)$, então $B \circ A \in K(X, Z)$.
    \item Se $A \in K(X, Y)$, então $A^* \in K(Y^*, X^*)$.
    \item Se $A \in K(X, Y)$ e $R(A)$ é um subespaço fechado de $Y$, então $R(A)$ tem dimensão finita.
\end{enumerate}
\end{theorem}
\end{multicols}

\end{comment}
\end{document}
