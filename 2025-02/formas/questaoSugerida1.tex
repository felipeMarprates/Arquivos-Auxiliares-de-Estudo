\documentclass[11pt,a4paper]{article}

% Pacotes básicos
\usepackage[utf8]{inputenc}
\usepackage[T1]{fontenc}
\usepackage{amsmath,amssymb,amsthm}

% Comando para o sinal da permutação
\newcommand{\sgn}{\varepsilon}

% Ambientes de teorema/prova
\newtheorem{teorema}{Teorema}
\newtheorem{proposicao}{Proposição}
\newtheorem{lema}{Lema}
\newtheorem{corolario}{Corolário}

\theoremstyle{definition}
\newtheorem{definicao}{Definição}

\theoremstyle{remark}
\newtheorem*{obs}{Observação}

\begin{document}

\section*{Exercício sugerido pelo grupo 3}

Prove que o produto tensorial
de uma forma $f \in N_p(E;\mathbb{R})$ por qualquer $g \in L_q(E;\mathbb{R})$
é uma $(p+q)$-forma, $f \cdot g \in N_{p+q}(E;\mathbb{R})$. Onde $N_p(E;\mathbb{R})$ 
é o núcleo do operador anti-simétrico $A$. 
\subsection*{demonstração:}
Seja  $f \in N_p(E;\mathbb{R})$ e $g \in L_q(E;\mathbb{R})$.\newline
Note que, basta mostrar que $ A(Af \otimes g) = {p!} A(f \otimes g)$. Pois, $A(f)=0$,
por hipótese, logo $(Af \otimes g)=0$. Dessa forma, é fácil ver que $A(0)=0$. Então,
se a igualdade vale, 


\[
 0 = A(Af \otimes g) =  {p!} A(f \otimes g) \implies f \otimes g \in N_{p+q}(E;\mathbb{R})
\]

Vamos mostrar que a igualdade vale. Temos que:
\begin{align*}
A(Af \otimes g) 
&= \sum_{\sigma \in S_{p+q}} \sgn_{\sigma}\sigma
\Bigg[ (\sum_{\phi \in S_p} \sgn_{\phi} \phi f)\otimes g \Bigg].
\end{align*}

Podemos, sem comprometer a expressão, enxergar $S_p$ como um subgrupo de $S_{p+q}$ 
que fixa as $q$ últimas coordenadas. Assim,
\begin{align*}
A(Af \otimes g) 
&= \sum_{\sigma \in S_{p+q}} \sgn_{\sigma}\sigma 
\Bigg[ \sum_{\phi \in S_p < S_{p+q}}  \sgn_{\phi }\phi (f \otimes g) \Bigg].
\end{align*}

Logo,
\begin{align*}
A(Af \otimes g) 
&= \sum_{\sigma \in S_{p+q}} 
\Bigg[ \sum_{\phi \in S_p<S_{p+q}} \sgn_{\sigma} \sgn_{\phi }(\sigma \circ \phi) (f \otimes g) \Bigg].
\end{align*}

Temos que, para todo $\sigma \in S_{p+q}$, existe um único $\lambda \in S_{p+q}$ tal que 
$\sigma = \lambda \circ \phi^{-1}$ . Assim,
\begin{align*}
A(Af \otimes g) 
&= \sum_{\lambda \in S_{p+q}} \sum_{\phi \in S_q<S_{p+q}} 
\sgn_{\lambda} \sgn_{\phi^{-1}} \sgn_{\phi} (\lambda \circ \phi^{-1} \circ \phi)(f \otimes g).
\end{align*}
Como o sinal da permutação vezes o sinal da permutação inversa é a identidade, Podemos simplificar:
\begin{align*}
A(Af \otimes g) 
&= \sum_{\lambda \in S_{p+q}} \sum_{\phi \in S_q<S_{p+q}} 
\sgn_{\lambda}  (\lambda)(f \otimes g).
\end{align*}
Portanto,
\begin{align*}
A(Af \otimes g)
&= p! \sum_{\lambda \in S_{p+q}} \sgn_{\lambda}\lambda (f \otimes g) \\
&= p! \, A(f \otimes g).
\end{align*}



\end{document}
